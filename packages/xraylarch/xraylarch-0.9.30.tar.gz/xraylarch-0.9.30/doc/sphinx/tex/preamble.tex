\submitted{November 2012}
\copyrightyear{2012}
\adviser{Professor John P. Smith}
\department{Computer Science}

\newcommand{\proquestmode}{}

\abstract{
\input{abstract.inc}
}

\acknowledgements{
\chapter*{Acknowledgements}
\addcontentsline{toc}{chapter}{Acknowledgements}

\section*{Institutions}

We thank Columbia University along with the Departments of
Statistics and Political Science, the Applied Statistics Center, the
Institute for Social and Economic Research and Policy ({\sc iserp}),
and the Core Research Computing Facility.

\section*{Grants and Corporate Support}

Without the following grant and consulting support, Stan would not exist.

\subsection*{Current Grants}

\begin{itemize}
\item 
 U.~S.\ Department of Education Institute of Education Sciences
\begin{itemize}\small
\item Statistical and Research Methodology: Solving
Difficult Bayesian Computation Problems in Education Research Using Stan
\end{itemize}
\item  Alfred P.\ Sloan Foundation
\begin{itemize}\small
\item G-2015-13987: Stan Community and Continuity (non-research)
\end{itemize}
\item U.~S.\ Office of Naval Research (ONR)
\begin{itemize}\small
\item Informative Priors for Bayesian Inference and Regularization
\end{itemize}
\end{itemize}
%


\subsection*{Previous Grants}

Stan was supported in part by
%
\begin{itemize}
\item
U.~S.\ Department of Energy 
\begin{itemize}\small
\item DE-SC0002099: Petascale Computing
\end{itemize}
%
\item
U.~S.\ National Science Foundation 
\begin{itemize}\small
\item
ATM-0934516: Reconstructing Climate from Tree Ring Data
\item
CNS-1205516: Stan: Scalable Software for Bayesian Modeling
\end{itemize}
%
\item
U.~S.\ Department of Education Institute of Education Sciences 
\begin{itemize}\small
\item ED-GRANTS-032309-005:
Practical Tools for Multilevel Hierarchical Modeling in Education
 Research
\item R305D090006-09A: Practical Solutions for Missing Data
\end{itemize}
\item
U.~S.\ National Institutes of Health
\begin{itemize}
\item 1G20RR030893-01: Research Facility Improvement Grant
\end{itemize}
\end{itemize}


\section*{Stan Logo}

The original Stan logo was designed by Michael Malecki.  The current
logo is designed by Michael Betancourt, with special thanks to
Stephanie Mannheim (\url{http://www.stephaniemannheim.com/}) for
critical refinements.  The Stan logo is copyright 2015 Michael
Betancourt and released for use under the CC-BY ND 4.0 license (i.e.,
no derivative works allowed).


\section*{Individuals}

We thank John Salvatier for pointing us to automatic differentiation
and \HMC in the first place.  And a special thanks to Kristen van
Leuven (formerly of Columbia's ISERP) for help preparing our initial
grant proposals.

\subsection*{Code  and Doc Patches}

Thanks for bug reports, code patches, pull requests, and diagnostics
to:
Ethan Adams,
Avraham Adler,
Jeffrey Arnold, 
Jarret Barber, 
David R.~Blair,
Miguel de Val-Borro, 
Ross Boylan, 
Eric N.~Brown, 
Devin Caughey, 
Daniel Chen,
Ashley Ford,
Jan Gl\"ascher,
Robert J.\ Goedman,
Danny Goldstein,
Tom Haber,
B.~Harris,
Kevin Van Horn,
Andrew Hunter,
Bobby Jacob,
Bruno Jacobs,
Filip Krynicki
Dan Lakeland,
Devin Leopold,
Nathanael I.~Lichti,
Titus van der Malsburg,
P.~D.~Metcalfe,
Linas Mockus,
Jeffrey Oldham,
Tomi Peltola,
Joerg Rings,
Cody T.\ Ross,
Patrick Snape,
Matthew Spencer,
Alexey Stukalov,
Fernando H.~Toledo,
Arseniy Tsipenyuk,
Zhenming Su,
Matius Simkovic,
Matthew Zeigenfuse,
and
Alex Zvoleff.

Thanks for documentation bug reports and patches to:
Avraham Adler,
Jeffrey Arnold,
Asim,
Jarret Barber,
Ryan Batt,
Frederik Beaujean,
Guido Biele,
Luca Billi,
Chris Black,
botanize (GitHub handle)
Arthur Breitman,
Eric C.~Brown,
Juan Sebasti\'an Casallas,
Daniel Chen,
Andy Choi, 
David Chudzicki,
Andria Dawson,
Jos\'{e} Rojas Echenique,
Andrew Ellis,
G\"{o}k\c{c}en Eraslan,
Rick Farouni,
Avi Feller,
Seth Flaxman,
Wayne Folta,
Ashley Ford,
Kyle Foreman,
Mauricio Garnier-Villarreal,
Christopher Gandrud,
Jonathan Gilligan,
John Hall,
David Hallvig,
Cody James Horst,
Herra Huu,
Bobby Jacob,
Max Joseph,
Julian King,
Fr\"anzi Korner-Nievergelt,
Takahiro Kubo,
Mike Lawrence,
Louis Luangkesorn,
Stefano Mangiola,
David Manheim,
Dieter Menne,
Evelyn Mitchell,
Sunil Nandihalli,
Eric Novik,
Julia Palacios,
Tamas Papp, 
Tomi Peltola,
Andre Pfeuffer,
Sergio Polini,
Sean O'Riordain,
Brendan Rocks,
Cody Ross,
Mike Ross,
Tony Rossini,
Nathan Sanders,
James Savage,
Terrance Savitsky,
Dan Schrage,
seldomworks (GitHub handle),
Janne Sinkkonen,
Yannick Spill,
Dan Stowell,
Alexey Stukalov,
Dougal Sutherland,
John Sutton,
Maciej Swat,
Andrew J.~Tanentzap,
Shravan Vashisth,
Aki Vehtari,
Damjan Vukcevic,
Matt Wand,
Amos Waterland,
Sebastian Weber,
Sam Weiss,
Luke Wiklendt,
Howard Zail, and
Jon Zelner.

Thanks to Kevin van Horn for install instructions for Cygwin and to
Kyle Foreman for instructions on using the MKL compiler.


\subsection*{Bug Reports}

We're really thankful to everyone who's had the patience to try
to get Stan working and reported bugs.  All the gory details are
available from Stan's issue tracker at the following URL.
%
\begin{quote}
\url{https://github.com/stan-dev/stan/issues}
\end{quote}




\vfill
\begin{center}
\hfill
\begin{minipage}[b]{2in}
  \footnotesize {\it Stanislaw Ulam, namesake of \Stan and co-inventor
    of Monte Carlo methods \citep{MetropolisUlam:1949}, shown here
    holding the Fermiac, Enrico Fermi's physical Monte Carlo simulator
    for neutron diffusion.}
  \\[3pt] \mbox{ } \hfill
  {\scriptsize Image from \citep{Giesler:2000}.}
\end{minipage} \ \ \ \ \
\begin{minipage}[b]{1.5in} \mbox{ } \hfill
  \includegraphics[width=1.5in]{img/ulam-fermiac.pdf}
\end{minipage}
\end{center}

}

\dedication{
\input{dedication.inc}
}

%%%%%%%%%%%%%%%%%%%%%%%%%%%%%%%%%%%%%%%%%%%%%%%%%%%%%%%%%%%%%\
%%%% Tweak float placements
% From: http://mintaka.sdsu.edu/GF/bibliog/latex/floats.html "Controlling LaTeX Floats"
% and based on: http://www.tex.ac.uk/cgi-bin/texfaq2html?label=floats
% LaTeX defaults listed at: http://people.cs.uu.nl/piet/floats/node1.html

% Alter some LaTeX defaults for better treatment of figures:
    % See p.105 of "TeX Unbound" for suggested values.
    % See pp. 199-200 of Lamport's "LaTeX" book for details.
    %   General parameters, for ALL pages:
    \renewcommand{\topfraction}{0.85}   % max fraction of floats at top
    \renewcommand{\bottomfraction}{0.6} % max fraction of floats at bottom
    %   Parameters for TEXT pages (not float pages):
    \setcounter{topnumber}{2}
    \setcounter{bottomnumber}{2}
    \setcounter{totalnumber}{4}     % 2 may work better
    \setcounter{dbltopnumber}{2}    % for 2-column pages
    \renewcommand{\dbltopfraction}{0.66}    % fit big float above 2-col. text
    \renewcommand{\textfraction}{0.15}  % allow minimal text w. figs
    %   Parameters for FLOAT pages (not text pages):
    \renewcommand{\floatpagefraction}{0.66} % require fuller float pages
    % N.B.: floatpagefraction MUST be less than topfraction !!
    \renewcommand{\dblfloatpagefraction}{0.66}  % require fuller float pages


%%%%%%%%%%%%%%%%%%%%%%%%%%%%%%%%%%%%%%%%%%%%%%%%%%%%%%%%%%%%%\
%%%% Use packages

%\usepackage{amsfonts}

%%% For figures
\usepackage{graphicx}
%\usepackage{subfig,rotate}

%%% for comments
\usepackage{verbatim}

% nice references
\usepackage[sort,numbers]{natbib}

% nice description style
\usepackage{enumitem}
\setlist[description]{style=nextline,
                      leftmargin=1.5em,
                      labelindent=1em,
                      topsep=1em,
                      itemsep=0em}

% for subfigure
\usepackage[margin=8pt]{subcaption}
\captionsetup{labelfont=bf}

% for code listings
\usepackage{listings}
\IfFileExists{inconsolata.sty}{\usepackage{inconsolata}}{\usepackage{zi4}}
\usepackage{color}
\definecolor{codebg}{RGB}{248,248,248} % mimics html code style
\definecolor{codeborder}{RGB}{204,204,204}
\lstset{breaklines=true,
        breakatwhitespace=false,
        basicstyle=\singlespacing\footnotesize\ttfamily,
        columns=fixed,
        frame=single,
        rulecolor=\color{codeborder},
        backgroundcolor=\color{codebg}
        }

%%% For tables
\usepackage{multirow}
% Longtable lets you have tables that span multiple pages.
\usepackage{longtable}

% Booktabs produces far nicer tables than the standard LaTeX tables.
%   see: http://en.wikibooks.org/wiki/LaTeX/Tables
\usepackage{booktabs}

% adds more formatting options to tables
\usepackage{array}

%set parameters for longtable:
% default caption width is 4in for longtable, but wider for normal tables
\setlength{\LTcapwidth}{\textwidth}

%%%%%%%%%%%%%%%%%%%%%%%%%%%%%%%%%%%%%%%%%%%%%%%%%%%%%%%%%%
%%% Printed vs. online formatting
\ifdefined\printmode

% Printed copy
% url package understands urls (with proper line-breaks) without hyperlinking them
\usepackage{url}


\else

\ifdefined\proquestmode
%ProQuest copy -- http://www.princeton.edu/~mudd/thesis/Submissionguide.pdf

% ProQuest requires a double spaced version (set previously). They will take an electronic copy, so we want links in the pdf, but also copies may be printed or made into microfilm in black and white, so we want outlined links instead of colored links.
\usepackage{hyperref}
\hypersetup{bookmarksnumbered}

% copy the already-set title and author to use in the pdf properties
\makeatletter
\hypersetup{pdftitle=\@title,pdfauthor=\@author}
\makeatother

\else
% Online copy

% adds internal linked references, pdf bookmarks, etc

% turn all references and citations into hyperlinks:
%  -- not for printed copies
% -- automatically includes url package
% options:
%   colorlinks makes links by coloring the text instead of putting a rectangle around the text.
\usepackage{hyperref}
\hypersetup{colorlinks,bookmarksnumbered}

% copy the already-set title and author to use in the pdf properties
\makeatletter
\hypersetup{pdftitle=\@title,pdfauthor=\@author}
\makeatother

% make the page number rather than the text be the link for ToC entries
%\hypersetup{linktocpage}
\fi % proquest or online formatting
\fi % printed or online formatting
